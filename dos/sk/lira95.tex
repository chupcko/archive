\documentclass{article}
\usepackage{alltt,array,theorem,xspace,multicol,tabularx}
\title{Implementation of Optimized SK Reduction Machine}
\author{Sa\v sa Malkov$^1$, Goran Lazi\' c$^2$, Aleksandra Ga\v cevi\'
c$^3$, Nenad Miti\' c$^2$
\vspace{1ex}\\
$^1$Department of Genetics, Institute for Biological Research,\\ 29.
novembra  142, Belgrade\\
$^2$Faculty of Mathematics, Studentski trg 16, Belgrade\\
$^3$ Faculty of Civil Engineering, Bulevar Revolucije 73, Belgrade
\vspace{1ex}\\
email:\{xpmfc08, xpmfc09, xpmfc16, xpmfm23\}@yubgss21.bg.ac.yu}
\date{}
\def \sp{\vspace{1ex}\\}
\def \spd{\vspace{2ex}\\}
\def \ll{Lispkit Lisp\xspace}
\def \ls{Lispkit/SQL\xspace}
\setlength {\textheight}{20.5cm}
\setlength {\textwidth}{14cm}
\pagestyle{empty}
\begin{document}
\vspace{7cm}
\maketitle
\thispagestyle{empty}
\noindent
\begin{abstract}
Every applicative evaluator implementation of a functional language
at procedural host system has to confront with efficiency. Lazy SK
reduction machine implementation has much more efficiency problems
than most others, because of its very low level and laziness. In order
to make fast applicative evaluator implementation with lower memory
demands, we have tried to improve performances of SK reduction machine
implementation. In this paper we describe some of optimization methods
we found, which can improve SK reduction machine as well as some other
evaluator implementations.
\end{abstract}
AMS Subj. Class.(1991): 68N99
\section{SK reduction machine}
An algorithm formalization is based on the idea that every object
involved in the evaluation process is operator-like object, called
combinator. All combinators can be defined using just few basic
combinators \cite{combilog}, and we use a formalization with two basic
combinators S and K. S is defined as three arguments operator, and its
application can be described as Sxyz$\mapsto$xz(yz), where parenthesis
denote a separate application. K is defined as two arguments operator
which selects the first argument: Kxy$\mapsto$x.
\sp
SK reduction machine is calculator for SK combinators based expressions.
SK combinators expressions have standard term syntax with two atoms:
S and K. In addition, even the set of constants can be defined using
S and K, it is added to the combinators as a language extension. In
this paper we introduce only natural numbers as constants. Complete
description of SK reduction machine can be found in \cite{funcprog},
\cite{implet}.
\section{Implementation problems}
If we take a look at implementations of many different functional
languages at usual procedural environments, we can note that many
problems encountered in some of these implementations are present in
SK reduction machine realization too. Lazy SK reduction machine becomes
an environment for efficiency testing, and if we develop some
optimization techniques for better SK reduction results, they could be
used in other implementations. We were interested in the next problems:
\begin{itemize}
\item
evaluation algorithms, that makes a heart of the reduction machine,
are recursive with unpredictable depth, which is a problem if a host
is a procedural system with limited stack size
\item
with normal order reduction, there are two classical redex searching
techniques: spine stack technique, which has unpredictable memory
requests, and pointer-reversal technique, which uses very little extra
storage, but is a bit more complex and slower than the first one
\item
recursion implemented by Y combinator is inefficient because of large
memory requests, and requires complex recursive combinators coding
\item
the previous problem extremely enlarges in presence of the mutual
recursiveness of many combinators
\item
garbage collector is never good enough (it is pragmatic question,
at least)
\end{itemize}
\section{Optimizations}
In order to find some better solutions for mentioned problems we have
designed the optimized SK reduction machine. We have started with no
optimization and included one by one optimization creating more
efficient machine.
\begin{enumerate}
\item
{\bf The recursiveness is removed from the implementation.}\\
All applied algorithms are non-recursive, including even the redex
searching, evaluation and memory freeing. Without recursiveness the
implementation at procedural host system has predictable and limited
memory requests for implementation itself. The implementation without
recursiveness is faster, too.
\item
{\bf New redex searching algorithm is based on a cyclic stack.}\\
This method is based on the fast redex searching algorithm using a
stack, and requires a limited memory amount. While operands data
fills the memory reserved for stack, algorithm chooses which nodes
to put on the stack, and which to remove from it. At evaluation time
it is necessary to partially reconstruct the stack which slightly
slows total evaluation process in relation to simple stack technique.
\item
{\bf Efficient SK recursiveness is implemented.}\\
Instead of use of Y combinator this implementation allows direct
notations of recursive functions, like F=\ldots F\ldots It is much
easier to code recursive combinators now, since their definitions
become much simpler and shorter with less operand shiftings through
the expression. Main feature this approach obtains is a faster
evaluation with less memory requirement. Direct notation of mutual
recursive expressions is allowed, too.
\item
{\bf Y is not included as system combinator.}\\
Y is excluded from system combinators in attempt to daunt its use
instead of direct notation of recursive expressions.
\item
{\bf No need for separate garbage collector.}\\
Because of absence of the cyclic Y combinator there is no cyclic
structures in evaluation trees at all, so reference count algorithm
for garbage collecting is sufficient.
\item
{\bf It is easy to add new system combinators.}\\
Internal reduction machine structure allows simple adding of new
system operators in order to create more complex machine with simpler
expressions.
\end{enumerate}
\section{Implementation}
The SK reduction machine is implemented in C language
according to its description given in
\cite{funcprog}, \cite{implet}. Reduction machine consists of four main
parts: memory manager, translator, evaluator and the module that
implements system operators.
\subsection{Memory manager}
Memory manager provides memory pool that contains cells. Cell is an
atomic memory unit used for SK expression and evaluation. Cell consists
of left and right parts, appropriate type information and reference
counter.
%Its structure is shown below (in C notation):
%\begin{alltt}
%      struct cell_struct {
%         char lefttype;
%         char righttype;
%         struct cell_struct *left;
%         struct cell_struct *right;
%         unsigned refcount;
%      };
%      typedef struct cell_struct Cell;
%\end{alltt}
Type can be: {\tt EMPTY\_TYPE}, {\tt POINTER\_TO\_CELL}, {\tt NUMBER},
{\tt SYSTEM\_OPERATOR}, {\tt USER\_OPERATOR}. Every cell describes an
application of operator at left side to the operand at right side. If
there is not any operand at right side (rigthtype value is
{\tt EMPTY\_TYPE}), then the result of application is operator itself.
Application {\tt(add 43 25)} will consist of two cells and could be
noted as application of {\tt(add 43)} combinator to {\tt 25}
(see Figure \ref{fig:add}).
%\begin{center}
\begin{figure}[h]
\setlength{\unitlength}{1mm}
\begin{picture}(140,15)
\put(52.5,0){\framebox(15,5){}}
\put(52.5,4){\line(1,0){15}}
\put(60,4){\line(0,-1){4}}
\put(52.5,0){\makebox(7.5,4){add}}
\put(60.0,0){\makebox(7.5,4){43}}

\put(65,10){\framebox(15,5){}}
\put(65,14){\line(1,0){15}}
\put(72.5,14){\line(0,-1){4}}
\put(65,10){\makebox(7.5,4){}}
\put(72.5,10){\makebox(7.5,4){25}}
\put(68,12){\vector(-1,-1){8}}
\end{picture}
\caption{Cells in application of {\tt (add 43)} combinator to {\tt 25}}
\label{fig:add}
\end{figure}
%\end{center}
Memory manager contains cell allocation and freeing algorithm,
together with a garbage collector. All free cells are organized in a
cell list from which are taken at allocation time, and returned after
freeing. Besides, there is an additional list that contains pointers
to trees that should be freed and is used for efficient non-recursive
garbage collecting. In both lists the right side of cell is used for
list linkage, and in additional list the left side is used to point
to unreferenced branch. Reference count garbage collector is
incorporated into the cell freeing algorithm. Simple cell freeing
algorithm is recursive, because cell can contain two pointers to cells:
same function is applied twice, once for each branch. Our freeing
algorithm is not recursive but cyclic, with determined memory demands.
If cell contains two pointers to other cells, one branch has to be
preserved in the list without allocating any additional cell, and
other has to be freed in the same way.
%(see Figure \ref{fig:alg}).
%\begin{figure}[h]
%{\baselineskip=0.80\baselineskip
%\begin{alltt}
%dereference:
%   if( --(cell_to_free->reference_count) == 0 )
%      if( cell_to_free->righttype == POINTER_TO_CELL ){
%         next_to_free = cell_to_free->right;
%         if( cell_to_free->lefttype == POINTER_TO_CELL ){
%            add_cell_to_list_of_branches( cell_to_free );
%            cell_to_free = next_to_free;
%         }else
%            add_cell_to_list_of_free( cell_to_free );
%         cell_to_free = next_to_free;
%         goto dereference;
%      }else if( cell_to_free->lefttype == POINTER_TO_CELL ){
%         next_to_free = cell_to_free->left;
%         add_cell_to_list_of_free( cell_to_free );
%         cell_to_free = next_to_free;
%         goto dereference;
%      }else
%         add_cell_to_list_of_free( cell_to_free );
%\end{alltt}
%}
%\caption{Freeing algorithm}
%\label{fig:alg}
%\end{figure}
When there is no cells in the list of free cells, memory manager
frees cells from the branch at the begin of the list of branches
that should be dereferenced.
\subsection{Translator}
Translator is a module that translates textual representation of
SK expressions to internal graph representation. There are two kind
of acceptable expressions: definition and evaluation. Definition
is expression that defines a new combinator (user operator), and
evaluation is an application of already defined combinators. Its syntax
is shown at the Figure \ref{fig:syntax}.
\begin{figure}[h]
%{\baselineskip=0.80\baselineskip
\begin{alltt}
<definition>   ::= <identifier> = <SKexpression>.
<evaluation>   ::= <SKexpression>.
<SKexpression> ::= <operator> [<operand>]
<operator>     ::= <identifier> | <constant> | <SKexpression>
<operand>      ::= <identifier> | <constant> | (<SKexpression>)
<combinator>   ::= <operator>   | <operand>
<constant>     ::= <integer>
<identifier>   ::= <character> [<any symbol different from `.`>]
\end{alltt}
%}
\caption{Syntax of acceptable expression in BNF form}
\label{fig:syntax}
\end{figure}
Some examples of user defined operators and evaluations are shown in the
example of application {\tt fact (sum 20)} on Figure \ref{fig:exampl}.
Comments are enclosed by {\tt\$\$}.
\begin{figure}[h]
%{\baselineskip=0.80\baselineskip
\begin{tabularx}{\linewidth}
{>{\setlength{\hsize}{1.44\hsize}\tt}X
 >{\setlength{\hsize}{0.56\hsize}\tt}X}
fact = S (S if (S mul (S (K fact) pred ))) (K 1).
&\$ fact n->n! \$\\
sum = S (S if (S add (S (K sum) pred ))) (K 0).
&\$ sum n->1+2+$\ldots$+n \$\\
mul2 = S (S(K add)I) I. &\$ mul2 x->x+x \$\\
mul2a = mul 2.  &\$ mul2a x->2*x \$\\
fib = S (C(B if(greater 2))1) (S (B add (B fib pred))
(B(B fib pred)pred)). &\$ fib n->(n-th \$ \mbox{\$ fibonacci num.)\$}\\
cons = S (S (KS) ( S(KK) (S(KS)(S(K(SI))K)))) (KK). &\\
%&\$ makes list of head \$ \mbox{\$ and tail \$}\\
head = S I (K K). &\\%&\$ head list->(head of list) \$\\
tail = S I (K(S K)).& \\% &\$ tail list->(tail of list) \$\\
fact (sum 20).& \\
\end{tabularx}
%}
\caption{Example of function definitions ans applications}
\label{fig:exampl}
\end{figure}
\subsection{Evaluator}
Evaluator searches the expression for the outermost redex and evaluates
it. It consists of the redex searching algorithm and the evaluation
distributor. We define a cyclic stack as a stack that can grow up to
the previously defined limit, and begin to overwrites itself after the
limit is reached. The stack notes how much values can be popped down
before it becomes invalid. In order to reach losed data, stack memorize
few key positions and builds the valid stack again from the nearest
known point when it is necessary. If rebuilding process become often,
the evaluation speed reduces, but fast redex searching algorithm can
compensate this. The stack algorithm provides correct evaluation even
if we limit the stack size to only four cells more than the maximal
number of operands for systems operators is. Expected slowness is not
really disturbing, because the redex searching is much simpler and
faster than evaluation of known redex and is optimised to use known
positions whenever it is possible. The evaluation distributor finds
out what kind of evaluation is appropriate for given operator. It
initiates the system operator evaluation or links the user operator
definition instead of its code. If a system operator has to evaluate,
the evaluation distributor consider when it is necessary to free
cells which are not used any more, so the system operator evaluation
function has not to worry about that. To make user operator definitions
linkage faster, it has to be done just once and only for the cells that
have to be applied. Evaluator signs when evaluation is done or system
resources come down and evaluation can't be finished. The description
of the evaluation algorithm is shown in the Figure \ref{fig:descr}.
\begin{figure}[ht]
%{\baselineskip=0.80\baselineskip
\begin{alltt}
function multiple_redex_evaluation( redex_root )
\{
   evaluating := true;
   cell := redex_root;
   while(evaluating)
     \{
      going_down := true;
      while(going_down)
        \{
         if (cell.right_tip != SKTYPE_EMPTY)
            add_to_argument_stack( cell );
         if (cell.left_tip == SKTYPE_USER_OPERATOR)
            \{
              cell.left_tip := SKTYPE_POINTER;
              cell.left := operator_definition_root(cell.left);
            \}
         if (cell.left_type = SKTYPE_POINTER)
            cell := cell.left;
            else going_down = false;
        \}
      if (cell.left_type == SKTYPE_SYSTEM_OPERATOR)
        \{
         arrity = operator_arrity( cell.left );
         strictness = operator_strictness( cell.left );
         if (enough_arguments_at_stack( arrity ))
            if strictness>0
              \{
               \ldots evaluate_strict_arguments\ldots
               cell := evaluate_system_operator( cell.left );
              \}
              else evaluating = false; /* an operator without arguments */
        \}
         else evaluating = false; /* a number instead of operator */
     \}
\}
\end{alltt}
%}
\caption{Evaluation algorithm - description}
\label{fig:descr}
\end{figure}
%     /* some stack manipulations is necessary to avoid recursiveness */
\subsection{The module that implements system operators}
Any system operator is defined by its evaluation function, arrity and
strictness. The operator arrity is a desired number of operands to
evaluate it, while strictness denotes how much of first few operands
have to be evaluated before the operator evaluation. In the pure
combinator language all operators are non-strict, and do not need any
operand to be evaluated before its evaluation, but when the language
is extended by some constants (numbers), then some basic operators on
these constants are necessary, and often their arguments have to be
evaluated. The nice example is addition: we can not evaluate
{\tt(add x y)} if we do not know what numbers x and y represent, so we
have to evaluate them first. Similarly, we can define conditional
operator and logical constants using only pure combinators, but it is
much simpler and faster to use numbers as logical constants and to
demand the first operand of the if-operator to be evaluated, so
if-operator has strictness 1. The basic system operators are only S
and K, and basic number operators are succ, pred and less. It is
obvious that more system operators will provide faster machine. We
have implemented if-operator, operators add and mul and few frequently
used pure combinators:
\vspace{2ex}\\
\begin{tabular}{lll}
I=SKK, Ix$\mapsto$x   &~~~~~~  & F=SK, Fxy$\mapsto$y \\
C=S(BBS)(KK), Cfxy$\mapsto$fyx & & B=S(KS)K, Bxyz$\mapsto$x(yz) \\
P=CI, Pxy$\mapsto$yx &\verb+      + & Y=S(CBD)(CBD), Yf$\mapsto$f(Yf)
\end{tabular}
\vspace{2ex}\\
After many tests, it was obvious that directly noted recursive user
operators are much faster techique with much smaller memory requests
than those using Y combinator in their definitions. To encourage the use
of direct recursive coding instead of use of the Y we have excluded Y
from the set of system operators.
\section{Conclusion}
This paper describes an implementation of optimized lazy SK reduction
machine. Because of low level and laziness of that machine, efficiency
problems arised. We have developed a more efficient machine using some
optimization techniques. Our SK reduction machine does not include any
kind of separate garbage collector, system defined Y combinator or
recursiveness in its procedural code, but it supports efficient
recursiveness in the SK expressions it evaluates. Main speed advance and
memory spare is obtained using evaluation algorithm that evaluates
recursive combinators without creating any cyclic structures in the
evaluation tree. Consequence of this speed improvement is that there is
no need for a separate garbage collector. Instead of it, a simple
reference count garbage collector is implemented.  Direct recursive
expression notation allows simpler combinators definitions which
evaluate faster.  The redex searching algorithm based on the cyclic
stack is a part of previously mentioned implementation without
recursiveness. The cyclic stack is better than the simple stack in the
use of memory but is a little slower when the stack rebuilding process
is necessary.
\begin{thebibliography}{9}
\bibitem{funcprog}
Field, A.J., and Harrison P.G.
{\em ``Functional programming''},
{\em Addison-Wesley}, 1989

\bibitem{combilog}
Curry H. B., and Feys R.
{\em Combinatory Logic. Vol 1.},
North-Holand 1958

\bibitem{implet}
Peyton Jones S. L.
{\em The Implementation of Functional Programming Languages},
Prentice Hall 1987
\end{thebibliography}
\end{document}
