\documentclass[12pt,a4paper,titlepage]{article}

  \def\d{d\kern-0.4em\char"16\kern-0.1em}
  \def\bs{$\backslash$}

  \title{\bf Programski jezik {\em AZOT}\\za rukovanje tekstom}
  \author{Goran Lazi\'c\\Matemati\v cki fakultet, Beograd}
  \date{1997. godina}

  \renewcommand\contentsname{Sadr\v zaj}
  \renewcommand\refname{Literatura}
  \newtheorem{primer}{Primer}[section]

\begin{document}
  \maketitle
  \tableofcontents
%
%
  \section*{Napomena, pre svega}
    U ovom radu se koriste razni termini koji su za\v sti\'ceni, bez posebnog
    nagla\v savanja o vlasni\v stvu nad imenom.
    Ovaj rad je prelomljen u sitemu \LaTeX2e.
%
%
  \newpage
  \section{Uvod}
    Sa pojavom ra\v cunara javlja se mogu\'cnost skladi\v stenja raznih
    informacija.
    Sa pove\'canjem brzine rada i pove\'canjem sme\v stajnih kapaciteta
    ra\v cunara, javlja se potreba za univerzalnim alatkama pogodnim za obradu
    informacija.
    Naime ve\'c je op\v ste prihva\'cen datote\v cni model skladi\v stenja
    informacija.
    Svi podaci (skladi\v steni u ra\v cunaru) se nalaze u datotekama koje su
    zapravo samo sekvencija (kona\v can niz) ograni\v cenih brojeva.
    Kori\v s\'cenjem raznih kodiranja, ljudski \v citljive informacije se
    prevode u sekvenciju brojeva, koja se zapisuje u datoteku.
    Datotekama mo\v zemo baratati (brisati, kopirati, menjati) zahvaljuju\'ci
    operativnom sistemu, koji podr\v zava rad nad datotekama.

    Op\v ste je prihva\'ceno da se datoteka sastoji od sekvencije okteta.
    Oktet je informacija koja se mo\v ze zapisati uz pomo\'c 8 bita, dakle
    neozna\v cen broj iz intervala od 0 do 255.
    Tako dolazimo do na\v cina zapisivanja informacija u ra\v cunaru.

    Primer kodiranja je kodiranje slike, gde se slika razgra\d uje na ta\v cke
    i svaku ta\v cku opisujemo kodom njene boje. Tako sliku prevodimo u
    sekvenciju brojeva. Na sli\v can na\v cin se kodira i zvuk. Dok kod
    kodiranja teksta imamo nekoliko problema.

    Sam tekst mozemo kodirati u sirovom formatu (\v cesto zvanom ASCII format
    teksta). U sirovom formatu re\d amo slova onako kako se pojavljuju uz
    jednu vrstu formatiranja.

    Za zapisisvanje teksta u sirovom formatu, koristi se sedmo bitni {\bf
    ASCII} (akronim od: {\em American Standard Code for Information
    Interchange}) ili osmo bitni {\bf EBCDIC} (akronim od: {\em Extend Binary
    Coded Decimal Interchange Code}).
    U ovim k\^odovima svakom grafi\v ckom znaku (koji se \v cesto koriste)
    pridodat je po jedan broj, njegov k\^od.
    {\bf ASCII} k\^od je zastupljeniji u trenutnim realizacijama operativnih
    sistema.

    Veoma je zanimljivo videti koji su sve grafi\v cki znaci zastupljeni u
    ovim k\^odovima.
    Zastupljena su slova engleske abecede (velika i mala), dekadne cifre,
    znaci interpunkcije, znaci osnovnih matemati\v ckih simbola kao i
    kontrolni znaci (koji se koriste za primitivno ure\d ivanje teksta):
    \begin{verbatim}
A B C D E F G H I J K L M N O P Q R S T U V W X Y Z
a b c d e f g h i j k l m n o p q r s t u v w x y z
0 1 2 3 4 5 6 7 8 9
. , : ; ' ` " ! ?
+ - / \ * = ^ < > ( ) [ ] { }
@ # $ % & _ | ~
\end{verbatim}
    Tu se jos nalazi i praznina (bez grafi\v ckog oblika), kojom odvajmo
    re\v ci (koju mo\v zemo ponekad posmatrati i kao kontrolni znak).
    A od kontrolnih znakova (koji imitiraju razne radnje koje smo mogli da
    proizvedemo na pisa\'cim ma\v sinama) imamo:
    \begin{itemize}
    \item
      Novi red, kojim prelazimo u novi red;
    \item
      Horizontalni tabulator, za horizontalno poravnanje;
    \item
      Vertikalni tabulator, za vertikalno poravnanje;
    \item
      Nova strana, kojom prelazimo na slede\'cu stranu;
    \item
      Zvonce, zvu\v cni signal.
    \end{itemize}

    Dakle tekst u ra\v cunaru se predtsavlja kao sekvencija ovih grafi\v ckih
    znakova i kontrolnih znakova.
    Ovi znaci su dovoljni samo za zapisivanje teksta koji smo mogli da
    otkucamo na pisa\'coj ma\v sini.
    Za svaku ve\'cu hijerarhijsku strukturu teksta neophodno nam je neko drugo
    kodiranje (koja vr\v simo nad {\bf ASCII} k\^odom).

    Tako recimo, program za neki programski jezik zapisujemo u formatu
    sirovog teksta. Rane konfiguracione datoteke za neki operativni sistem
    su u formatu sirovog teksta, kao i rezultati rada raznih komandi
    operativnog sistema. Za zapis grfi\v ckog oblika prirodno jezi\v ckog
    dokumenta mozemo koristiti recimo sisteme {\bf POSTSCRIPT}, {\TeX},
    {\LaTeX}, {\bf troff}. A za sam zapis prirodno jezi\v ckog dokumenta
    naprimer {\bf SGML} ili {\bf HTML} u kojima se u odgovaraju\'coj sintaksi
    zapisuje prirodno jezo\v cki dokumet u sirovom formatu teksta.

    U ovom radu posmatra\'cemo neke alatke za obradu teksta u sirovom
    formatu, kao i jedan novi programski jezik za rukovanje tekstom.
%
%
  \newpage
  \section{Rad sa znacima u {\bf UNIX}}
    U konkretnom operativnom sistemu {\bf UNIX}-u (opisanom u \cite{UNIX})
    datoteke koje sadr\v ze tekst su od velikog zna\v caja jer sva
    pode\v savanja, svi rezlutati rada su neke tekstualne datoteke.
    Zato se u koncepciji {\bf UNIX}-a nalazi i programski jezik {\bf C}
    (opisan u \cite{KR}) na kome mo\v zemo napisati program koji obra\d uje
    tekstualne datoteke.
    Tokom vremena izkristalisalo se nekoliko alatki koje se \v cesto koriste i
    one su izdvojene u posebne komande {\bf UNIX} operativnog sistema.

    Velika prednost operativnog sistema {\bf UNIX} je da se rezultat jedne
    komande mo\v ze proslediti drugoj komandi (takozvani: piping).
    Tako da re\d anjem vi\v se elementarnih komandi mo\v zemo uraditi neku
    slo\v zeniju operaciju.

    Sve o slede\'cim komandama mo\v ze se prona\'ci u \cite{UNIX} ili u man
    stranama\footnote{Upustvo, pomo\'c na {\bf UNIX} operativnom sistemu}
    (\cite{MAN}).
%
    \subsection{\em sort}
      Komandom {\bf sort} mo\v zemo sortirati redove tekstualne datoteke.
%
    \subsection{\em grep}
      \label{sub:grep}
      Komandom {\bf grep} pronalazimo sve redove neke datoteke koje sadr\v ze
      neki regularni izraz.
      Regularni izraz je opisan na slede\'ci na\v cin:
      \begin{itemize}
      \item
        Bilo koji znak, koji nema specijano zna\v cenje je regularni izraz.
        Ako \v zelim znak koji ima specijano zna\v cenje ipred njega stavimo
        {\tt\bs}, stim da slede\'ce sekvence ozna\v cavaju:
        \begin{itemize}
        \item
          {\tt\bs a} je zvonce;
        \item
          {\tt\bs b} je brisanje zadnjeg znaka (backspace);
        \item
          {\tt\bs f} je nova strana;
        \item
          {\tt\bs n} je novi red (newline);
        \item
          {\tt\bs r} je pomeranje kolica (carriage return);
        \item
          {\tt\bs t} je horizontalna tabulacija;
        \item
          {\tt\bs v} je vertikalna tabulacija;
        \item
          {\tt\bs ooo} je znak sa k\^odom {\tt ooo}, gde je {\tt o} oktalna
          cifra;
        \item
          \bs\bs\ je sam \bs.
        \end{itemize}
      \item
        Ta\v cka ({\tt.}) zamenjuje bilo koji znak osim kraja reda
        ({\tt\bs n}).
      \item
        Skup slova pi\v semo izme\d u {\tt[} i {\tt]} tako \v sto nabrojimo
        znakove za koje \v zelimo da pripadaju skupu.
        Opseg unutar skupa opsujemo crticom ({\tt-}) izme\d u prvog i
        poslednjeg znaka.
        Ako \v zelimo komplement skupa posle {\tt[} navodimo {\tt\^\ }.
      \item
        Konkatenaciju ne obele\v zavamo, ve\'c jednostavno spojimo regularne
        izraze.
      \item
        {\tt?} ozna\v cava opciono pojavljivanje regularnog izraza.
      \item
        {\tt*} ozna\v cava ponavljanje regularnog izraza, uklju\v cuju\'ci i
        0 puta.
      \item
        {\tt+} ozna\v cava ponavljanje regularnog izraza od 1 put i vi\v se.
      \item
        {\tt\{n\}} ozna\v cava ponavljanje regularnog izrzaza ta\v cno $n$
        puta.
      \item
        {\tt\{n,\}} ozna\v cava ponavljanje regularnog izrzaza bar $n$ puta.
      \item
        {\tt\{,n\}} ozna\v cava ponavljanje regularnog izrzaza najvi\v se $n$
        puta.
      \item
        {\tt\{n,m\}} ozna\v cava ponavljanje regularnog izrzaza izmdeju $n$ i
        $m$ puta.
      \item
        {\tt|} ozna\v cava opciju izme\d u dva regularna izraza.
      \item
        {\tt\^\ } ozna\v cava meta-znak koji obele\v zava po\v cetak reda.
      \item
        {\tt\$} oznza\v cava meta-znak koji obele\v zava kraj reda.
      \item
        Zagradama {\tt(} i {\tt)} mo\v zemo menjati prioritet operatora.
      \end{itemize}

      Na ovom mestu mo\v zemo primetiti da se koristi logi\v cka podela
      tekstualnih datoteka na redove (razdvojene znakom za kraj reda).
      Takva logi\v cka podela ne zadovoljava realne potrebe obrade teksta
      jer ne\'ce uvek logi\v cke celine biti razdvojene znakom za kraj reda.
%
    \subsection{\em uniq}
      Komanda {\bf uniq} zibacuje sve ponovljene redove (sortirane) tekstualne
      datoteke.
%
    \subsection{\em tr}
      Komandom {\bf tr} mo\v zemo neki znak promeniti u neki drugi znak,
      izbrisati sve pojave nekog znaka ili eliminisati uzastopna ponavljanja
      nekog znaka.
%
    \subsection{\em sed}
      Ako ve\'c i pristanemo na logi\v cku orkanizaciju po redovima, potrebna
      nam je alatka kojom mo\v zemo menjati sadr\v zaj tekstualne datoteke.
      Komanda {\bf sed} (akronim od stream editor) nam omogu\'cava da
      bri\v semo, dodajemo, menjamo redove neke tekstualne datoteke.

      Komandi {\bf sed} prenosimo instrukcije koja, zatim, ona obra\d uje.
      Instrukcije su oblika: {\it adresa[,adresa] funkcija[argumenti]}.

      Adresa je:
      \begin{itemize}
      \item
        Broj koji ozna\v cava broj reda;
      \item
        Regularni izraz (opisan u \ref{sub:grep});
      \item
        Znak {\tt\$} koji ozna\v cava poslednji red tekstualne datoteke.
      \end{itemize}

      Adresom odre\d umo dejstvo funkcije, ako nije navedena adresa tada je
      dejstvo funkcije cela tekstualna datoteka.

      Funkcije su:
      \begin{itemize}
      \item
        {\tt a\bs}\\
        {\it tekst}\\
        Dodajemo {\it tekst} posle adresirane linije, zahteva samo jednu
        adresu;
      \item
        {\tt c\bs}\\
        {\it tekst}\\
        Menjamo adresirano podru\v cje {\it tekstom};
      \item
        {\tt d}\\
        Bri\v semo adresirano podru\v cje;
      \item
        {\tt i\bs}\\
        {\it tekst}\\
        Dodajemo {\it tekst} pre adresirane linije, zahteva samo jednu adresu;
      \item
        {\tt p}\\
        Ispi\v se adresirano podru\v cje;
      \item
        {\tt r {\it datoteka}}\\
        Iskopira sadr\v zaj {\it dototeke} posle adrese, zahteva samo jednu
        adresu;
      \item
        {\tt s/{\it regularni-izraz}/{\it zamena}/{\it zastavica}}\\
        Menja u adresiranom podru\v cju prvu pojavu {\it regularnog-izraza}
        tekstom {\it zamene}.

        {\it regularni-izraz} je zapisan u skladu sa opisom u \ref{sub:grep}
        ali uz dodatak: delove regularnog izraza mo\v zemo ozna\v citi sa
        {\tt\bs(} i {\tt\bs)}.
        Naime tako ozna\v cene delove regularnog izraza mo\v zemo uvrstiti u
        {\it zamenu}.
        Svaka pojava {\tt\bs o} u tekstu zamene, gde je {\tt o} cifra, se
        menja $o$-tim ozn\v cenim delom regularnog izraza.
        Svaka pojava {\tt\&} u tekstu zamene se menja celim prepoznatim
        regularnim izrazom.

        {\it Zastavica} mo\v ze biti broj (tada se vr\v si menjanje samo na
        tom mestu) ili slovo {\tt g} (tada se vr\v se sve zamene)
      \item
        {\tt w {\it datoteka}}\\
        Snima adresirano podru\v cje u {\it datoteku};
      \item
        {\tt y/{\it niska-1}/{\it niska-2}/}\\
        Menja, u adresiranom podru\v cju, svaki znak iz {\it niske-1} u
        odgovaraju\'ci znak {\t niske-2};
      \item
        {\tt!{\it funkcija}}\\
        {\it Funkciju} izvr\v sava samo za redove koji ne pripadaju
        adresiranom prostoru.
      \end{itemize}

      \begin{primer}\ \label{primer:sed}\end{primer}
      \begin{verbatim}
/%/s/%\(.*\)$/\{\1\}/
/Swap(.*,.*)/s/Swap(\(.*\),\(.*\))/\2 \1/
/./!d
\end{verbatim}

      Ovaj {\bf sed} program radi slede\'ce:
      \begin{enumerate}
      \item
        U svakom redu koji sadr\v zi {\tt\%}, stavlja u viti\v caste zagrade
        {\tt\{} i {\tt\}} tekst iza {\tt\%} pa do kraja reda;
      \item
        U svakom redu koji sadr\v zi poziv ``funkcije'' {\tt Swap} okrene
        argumente funkcije {\tt Swap};
      \item
        Bri\v se sve redove koji ne sadr\v ze nijedan znak.
      \end{enumerate}
%
    \subsection{\em cut}
      Komandom {\bf cut} izdvajamo neko polje iz svkog reda neke tekstualne
      datoteke.
      Polja su deo reda koji su razdvojeni nekim, unapred definisanim, znakom.
%
    \subsection{\em paste}
      Komandom {\bf paste} spajamo dve tekstulane datoteke, red po red.
%
    \subsection{\em join}
      Komandom {\bf join} spajamo one redove dve tekstualne datoteke koje
      imaju isto neko polje.
%
%
  \newpage
  \begin{thebibliography}{}
  \bibitem{KR}
    B. W. Kernighan, D. M. Richi\\
    ``The C Programming Language''\\
    Prentice-Hall (1978)
  \bibitem{UNIX}
    ``UNIX System V Relase 2.0''\\
    Bell Laboratiories (1983)
  \bibitem{MAN}
    ``UNIX Programmer's Manual''\\
    Berkley Software Distribution (1991)
  \bibitem{AU}
    A. V. Aho, J. D. Ulman\\
    ``The Theory of Parsing, Translation and Compiling''\\
    Addison-Wesley (1978)
  \bibitem{LEX}
    M. E. Lesk\\
    ``LEX-A Lexical Analyser Generator''\\
    Bell Laboratiories (1975)
  \bibitem{YACC}
    Stephen C. Johnson\\
    ``Yet Another Compiler Compiler''\\
    Bell Laboratiories (1975)
  \bibitem{AGJP}
    Goran Lazi\'c\\
    ``Automatsko generisanje jezi\v ckih procesora''\\
    Matemati\v cki fakultet, Beograd (1994)
  \bibitem{AKS}
    Goran Lazi\'c\\
    ``Pribli\v zno upore\d ivanje niski''\\
    Matemati\v cki fakultet, Beograd (1996)
  \bibitem{AWK}
    A. V. Aho, B. W. Kernighan, P. J. Weinberger\\
    ``The AWK Programming Language''\\
    Addison-Wesley (1988)
  \bibitem{PERL}
    Larry Wall\\
    ``Learning PERL''\\
    O'Reilly \& Associates (1991)
  \end{thebibliography}
\end{document}
