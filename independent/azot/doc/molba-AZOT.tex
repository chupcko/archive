\documentclass[12pt,a4paper]{article}
  \def\d{d\kern-0.4em\char"16\kern-0.1em}
  \pagestyle{empty}
\begin{document}
  \parindent0pt
  {
    \Large{\bf MOLBA}
    \vskip 1cm
    Nastavno nau\v cnom ve\'cu Matemati\v ckog fakulteta
    \vskip 2mm
    Univerziteta u Beogradu
  }
  \vskip 1 cm
  Molim vas da mi odobrite izradu magistarskog rada.
  \vskip 5mm
  10.11.1995 godine upisan sam na studije za sticanje akademskog naziva
  magi\-stra nauka, na smeru ``Ra\v cunarstvo i informatika'', dosije broj
  9/95.
  \vskip 5mm
  Polo\v zio sam sve predmete predvi\d ene planom studija koje je Nastavo
  nau\v cno ve\'ce Matemati\v ckof fakulteta Univerziteta u Beogradu
  prihvatilo.
  \vskip 5mm
  U dogovoru sa profesorom dr. Du\v skom Vitasom odabrao sam temu:
  ``Programski jezik {\bf AZOT} za rukovanje tekstom''
  \vskip 5mm
  U prilogu se nalazi obrazlo\v zenje rada.
  \vskip 3cm
  \rightline{Goran Lazi\'c\hskip 2cm}

  \newpage

  \begin{center}
    \Large Obrazlo\v zenje rada:
    \vskip 2mm
    ``Programski jezik {\bf AZOT} za rukovanje tekstom''
  \end{center}
  \vskip 1 cm
  U uvodu rada se opisuje ideja o skladi\v stenju informacija u tekstualnim
  datotekama kao i odnos tekstualne datoteke i teksta. Zatim se obra\d uju
  pro\-blemi vezani za obradu teksta, kao i algoritmi obrade teksta.

  U radu se opisuju neke osnovne komande operativnog sistema {\bf UNIX}
  kojima se mo\v ze rukovati tekstom, kao i nedostatak takvog na\v cina
  rukovanja tekstom.

  Zatim se opisuju se programski jezici {\bf AWK} i {\bf PERL}, njihova
  koncepcija rukovanja tekstom, kao i nedostaci koji se javljaju.

  Opisuju se i neka zaglavlja sa korisnim funkcijama i neke alatke koje
  slu\v ze za generisanje specifi\v cnog programskog k\^oda (za programski
  jezik {\bf C}). Opisuju se: {\bf REGEXP}, {\bf LEX}, {\bf YACC},
  {\bf AKS}, {\bf AGLA}, {\bf AGSA}.

  Zatim se opisuje se sintaksa programskog jezika {\bf AZOT}, kako
  programski jezik {\bf AZOT} odogovara na probleme u rukovanju tekstom i
  kako se programski jezik {\bf AZOT} mo\v ze upotrebiti za sistemsko
  programiranje.

  U zaklju\v cku se opisuju lo\v se strane programskoj jezika {\bf AZOT},
  kao i \v sta je potrebno jo\v s uraditi.
  \vskip 5mm
\end{document}
